\section{Design}
\label{sec:design}
The high-level design goal of libhio is to provide a portable library
that will hide implementation-specific details of the burst buffer on
Trinity. The initial release of libhio will provide the following features:

\begin{itemize}
\item An iterface for querying if a checkpoint is recommended or
  required. This will provide applications with the ability to
  optimize their checkpoint frequency for the job size and current
  machine statistics (failure rates, BB status, etc). Future versions
  may hook into the resource manager to gather additional information
  to optimize checkpoint speeds.
\item Provide high-level APIs to for creating data sets and
  files. More information on these concepts can be found in
  \ref{sec:dataset} and \ref{sec:files}.
\item Provide high-level APIs for reading and writing contiguous and
  strided data.
\item Support multiple backends. These backends and their relative
  priorities will be configurable at run-time by the end
  user. Backends will include (but not be limited to): SCR
  (memory-backed), PFS (parallel file system), and BB.
\item Support transparently falling back on another backend when
  the requested backend fails. Examples include falling back on the
  PFS on detection of burst buffer failure.
\item Provide an interface to provide filesystems hints and
  configuration options. This interface will ideally include support
  for both per-file and per-context hints and options.
\item Provide a flexible configuration interface that to allow libhio
  options to be set by environment, file, or API calls.
\end{itemize}
